% Before processing this TeX file, you must load the TeX color commands from
% one of the '02*-*tex-import-color' files. For the best results, this file
% should be input after all other packages. The macros for coloring citations 
% are just instrumented macros from LaTeX and hyperref. If hyperref is 
% imported after this file, coloring of citations will be disabled because the 
% the new import will overwrite the colorized versions of hyperref macros.

% '@' needs to be processed in command names so we can access lower-level macros
% (e.g., '\current@color', '\Gin@driver') that will be defined by the TeX engines
% to let us know what's doing the coloring, and to let us set colors.
\catcode`@ = 11

\scholarifdefined{current@color}{%
%
% Only define the new color macros if they haven't yet been defined. This is
% important, because we want to make sure that these macro files can be included
% in the same TeX project multiple times without failure (i.e. for example, in
% all TeX files if we don't know which is the 'main' file).
\scholarifundefined{scholarsetcolor}{%
%
% ===============
% scholarsetcolor
% ===============
% Set color for everything in the document after this command.
% Color will apply even after the current group is finished.
%
\def\scholarsetcolor[#1]#2{%
{\csname color@#1\endcsname\current@color{#2}%
\ifx\Gin@driver\pdftexdrivername%
\pdfcolorstack0 push {\current@color}%
\else\ifx\Gin@driver\dvipsdrivername%
\special{color push \current@color}%
\else\message{Coloring not implemented for driver \Gin@driver}%
\fi\fi%
}%
}%
%
% ==================
% scholarrevertcolor
% ==================
% Revert a color set in a 'scholarsetcolor' command.
%
\def\scholarrevertcolor{%
\ifx\Gin@driver\pdftexdrivername%
\pdfcolorstack0 pop%
\else\ifx\Gin@driver\dvipsdrivername%
\special{color pop}%
\else\message{Coloring not implemented for driver \Gin@driver}%
\fi\fi%
}%
%
% ========================
% scholarregistercitecolor
% ========================
% Register a color for a citation key. Everywhere a source is cited using this
% key, the citation body will appear in this color. Takes four arguments:
% citation key, red, green, blue. Take care not to register the same color
% multiple times; it is not guaranteed this macro will handle it in a sane way.
%
\def\scholarregistercitecolor#1#2#3#4{%
\expandafter\def\csname scholarcolor@#1\endcsname{#2,#3,#4}%
\definecolor{scholarcolor@#1}{rgb}{#2,#3,#4}%
}%
%
% =======================================
% Rules for dynamic coloring of citations
% =======================================
% It's important that these rules are defined only once. For some TeX engines,
% if they are defined multiple times, it will result in infinite macro expansion.
%
% == scholar@color@if@citation@registered == 
% * This command colors text with a color registered for the key, if a color
% * has been registered for that key. The first parameter is the citation key.
% * In many (but not all) cases, you can supply the \@citeb for this parameter,
% * as LaTeX often sets this value as it's formatting a citation. The second
% * parameter is the TeX you want to color.
\def\scholar@color@if@citation@registered#1#2{%
\scholarifdefinedelse{scholarcolor@#1}{%
\textcolor{scholarcolor@#1}{#2}%
}{#2}%
}%
%
% == scholar@extract@citation@key ==
% * Extract a citation key from hyperref macro "hyper@link@" argument #3.
% * The argument is usually of the form 'cite.<key>', so remove the first
% * 5 characters to get the key.
\def\scholar@extract@citation@key #1#2#3#4#5{}%
%
% == Instrument default LaTeX citation formatting ==
% * Insert a color command when LaTeX is formatting a known citation.
\let\scholar@inner@cite@ofmt\@cite@ofmt
\def\@cite@ofmt#1{\scholar@color@if@citation@registered{\@citeb}{%
\scholar@inner@cite@ofmt{#1}%
}}%
%
% == Instrument hyperref citation formatting macros ==
% * Fetch the key of the citation that's currently being formatted. If no
% * citation key can be found, return empty.
\def\scholar@current@citation@key{%
% Check whether @citeb (which typically contains the citation key used
% to look up citation color) is defined. This requires multiple checks.
% First, check to see if @citeb was set to the '@nil' special value.
% Make sure @nnil is defined, as it will only be defined for LaTeX.
\scholarifdefinedelse{@nnil}{%
\ifx\@citeb\@nnil%
\scholar@key@from@Hy@testname%
\else%
\scholar@key@from@citeb@or@Hy@testname%
\fi%
}{%
\scholar@key@from@citeb@or@Hy@testname%
}}%
% * Helpers for extracting citation keys from active macros in the environment.
\def\scholar@key@from@Hy@testname{%
\scholarifdefined{Hy@testname}{%
\expandafter\scholar@extract@citation@key\Hy@testname%
}}%
\def\scholar@key@from@citeb@or@Hy@testname{%
\scholarifdefinedelse{@citeb}{\@citeb}{\scholar@key@from@Hy@testname}%
}%
% * In a previous version, we instrumented '\@citecolor' instead of
% * '\hyper@link', but this didn't always work. One hypothesis is that
% * '\@citecolor' wasn't getting called if 'colorlinks' was set to false.
\let\scholar@inner@hyper@link@\hyper@link@
\def\hyper@link@[#1]#2#3#4{%
\scholar@inner@hyper@link@[#1]{#2}{#3}{%
%
% Only attempt to color if this hyperlink is for a citation.
\def\scholar@link@type{#1}%
\def\scholar@link@citation@type{cite}%
\ifx\scholar@link@type\scholar@link@citation@type%
\edef\scholar@citation@key{\scholar@current@citation@key}%
\ifx\scholar@citation@key\empty%
\edef\scholar@citation@key{\scholar@extract@citation@key #3}%
\fi%
% Color the citation text.
\scholar@color@if@citation@registered{\scholar@citation@key}{#4}%
\else%
% This part will be executed if it's not a citation hyperlink.
#4%
\fi%
}}%
%
% == Instrument natbib citation formatting macros ==
% * The wrapped macros are defined in 'natbib.sty' in the TeXLive distribution.
\let\scholar@inner@NAT@hyper@\NAT@hyper@
\def\NAT@hyper@#1{\scholar@color@if@citation@registered{%
\scholar@current@citation@key%
}{%
\scholar@inner@NAT@hyper@{#1}%
}}%
% To fool-proof coloring for natbib, the \@citecolor macro from hyperref is
% also redefined to return the registered color, as the color from \@citecolor
% takes precedence over the color from \NAT@hyper. \@citecolor will be
% defined if 'colorlinks' is set to true in \hypersetup. The \@citecolor macro
% is redefined both as this macro is loading, and whenever \hypersetup is called
% in the middle of a TeX file (which redefines \@citecolor by default).
\def\scholar@wrap@citecolor{%
% Only define inner citecolor macro once. Because this \scholar@wrap@citecolor setup
% macro can (and will) be called multiple times, defining the inner citecolor command
% to be the citecolor command that wraps it would cause a recursive loop.
\scholarifundefined{scholar@inner@citecolor}{%
\let\scholar@inner@citecolor\@citecolor
}%
\def\@citecolor{%
\scholarifdefinedelse{scholarcolor@\scholar@current@citation@key}{%
scholarcolor@\scholar@current@citation@key%
}{%
\scholar@inner@citecolor%
}}%
}%
\scholar@wrap@citecolor{}%
\scholarifdefined{hypersetup}{%
\let\scholar@inner@hypersetup\hypersetup%
\def\hypersetup#1{%
\scholar@inner@hypersetup{#1}%
\scholar@wrap@citecolor{}%
}}%
\message{Defined S2 LaTeX coloring commands.}%
}}%

% Revert '@' to just be a normal character.
\catcode`@ = 12

\newcount\initialpageno
\newcount\currentpageno

\def\scholaradvancepage{
\message{Advancing to the next page. Starting on page \the\count0. }
\initialpageno = \count0
\loop
  \message{Still on page \the\count0. Forcing end of column. }
  \ 
  \eject
\ifnum \count0=\initialpageno
\repeat
\message{Finished advancing. Now on page \the\count0. }
}

\def\scholarfillandadvancepage{
\message{Adding a huge solid color box to page \the\count0. }
{\color[RGB]{250,165,80}\hrule width5000pt height5000pt}
\scholaradvancepage{}
}