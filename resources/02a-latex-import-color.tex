% Assumes '01-macros.tex' has already been loaded.
% This file should be included only at the end of the preamble, so that
% it can check to see whether the color commands were already loaded. If so, we 
% we don't want to load them a second time.

% Make sure the color commands are initialized. Only run these initializations 
% if the color commands haven't already been defined.
\scholarifundefined{color}{%
\message{%
'color' command not already loaded, but it's required for colorization of
entities. Importing 'color' package.}%
% This import of 'color' often isn't really needed, because AutoTeX add 
% statements to the TeX for many papers to import 'hyperref', which
% requires 'color'. That said, we don't know if AutoTeX will always succeed at 
% importing hyperref, so we import 'color' just to make sure.
\usepackage{color}%
% If the 'dvips' driver is used to compile this TeX file, there's a risk to 
% importing the 'color' package: some color drivers redefine the default 'black'
% color to a slightly lighter black that what would have otherwise been used if 
% the 'color' package wasn't imported. This code attempts to revert the default 
% black to what it would have been had these macros not been defined. See the
% message below for more details.
\makeatletter%
% XXX(andrewhead): For some reason, at this point of TeX processing, \Gin@driver is not
% available as a macro. The workaround used here is to expand the Gin@driver 
% control sequence into a new macro, and check the value of that macro.
\edef\scholardrivername{\csname Gin@driver\endcsname}%
\ifx\scholardrivername\dvipsdrivername%
\message{%
This TeX file is getting compiled using the 'dvips' driver. Because the color 
package wasn't imported before, the default color is being set to black, defined 
as "rgb (0,0,0)", to be consistent with the default color if the 'color' package 
had never been imported. See https://tex.stackexchange.com/a/210088/198728 for a 
description of this issue and of the fix used here.
}%
\definecolor{black}{rgb}{0,0,0}\color{black}%
\fi%
\makeatother%
}
